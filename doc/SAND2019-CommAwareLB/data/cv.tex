We begin by recalling that,  given a population of $X$ (in our case
with finite cardinality $n=\lvert{X}\rvert$) with arithmetic mean
$\mu(X)\neq0$, its \emph{coefficient of variation} is
defined as follows:
\[
c_V(X) = \frac{\sigma_{n}(X)}{\mu(X)},
\]
where $\sigma_{n}(X)$ denotes the standard deviation of $X$
(uncorrected for bias as we do not assure normality, nor 
sub-sampling of $X$). 

The main rationale for the use of $c_V$ as a quantitative estimate of
persistence is manifold, in particular:
\begin{enumerate}
\item
$X$ contains values of variables expressed in the same \underline{units} (in particular, no relative units).
\item
It is expected that no constant type zero measurements will be observed.
\item
We are \underline{not} seeking to build confidence intervals.
\item
We are not looking for measurement assurance (in which case standard
error would be more suitable).
\end{enumerate}

Furthermore, in order to capture persistence over
$k\in\llbracket2;+\infty[$ successive simulation phases (or steps), we
compute this the coefficient of variation of the quantity of interest
(here, per-object times) across as sliding window of length $k$, as
$c_V(X_{i,-k})$ where $X_{i,-k}$ denotes 
the subset of values in $X$ with indices $i-k+1$ to $i$, where
$i\in\llbracket{k-1};+\infty[$ is the phase index of interest
(assuming that the initial phase corresponds to $i=0$).
