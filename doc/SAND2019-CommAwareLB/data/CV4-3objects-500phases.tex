This methodology is illustrated in
Figure~\ref{f-CV4-3objects-500phases}, where $c_V(X_i,4)$ is computed
for the times of three separate objects, at each phase $i>2$ of a
$500$-phase simulation: each of these plots illustrates that, after an
intial sharp variation due to cache warm-up and other initilization
transients, the time-windowed per-object coefficients of varaiation
constantly remain well below $10\%$.

From these plots one can readily conclude that the initial object time
values exhibit too much relative variation with respect to their
time-averaged baselines to allow for a successful attempt at
load-balancing. In contrast, it appears that during the remainder of
the simulation, these objects' respective times only vary mildly with
respect to the values observed in the three preceding phases.

However, possibly very many objects participate in the simulation, and
it is not feasible to examine each individual time-sliding coefficient
of variation. Instead, a practical measurement of overall object time
persistence is simply the arithmetic mean of all coefficients, for it
is sensitive to extreme values and, being bounded-below (by $0$) but
not above, this mean will readily respond to each large relative
variation. We thus propose, as a measure of overall per-object time
persistence, the following \emph{$k$-persistence coefficient} of
object times:
\[
\mathcal{P}_k(\textbf{O}) = \frac{1}{\lvert\textbf{O}\rvert}
\sum_{O\in\textbf{O}} c_V(\ell_{i,-k}(O)),
\]
where $\ell_{i,-k}(O)$ denotes the subset of times of an object $O$
from phases $i-k+1$ to $i$.
