An application of this $k$-persistence coefficient is illustrated in
Figure~\ref{f-3-4-10-persistence-500phases} for values of
$k$ in $\{3;,4;10\}$ where $\textbf{O}$ contains a total of 64 objects,
including those whose individual time coefficient of variations are
illustrated in Figure~\ref{f-CV4-3objects-500phases}.

We observe that as the length of the time-window increases, the
averaged relative variation increases but its own variability
decreases. However, overall, these different values of $k$ yield
qualitatively similar informations, as it is important to note that
the vertical axis is capped at $10\%$, in order to emphasize the low
overall relative variation past the first few phases: even the spikes
in Figure~\ref{f-3-4-10-persistence-500phases} remain below the rate
of $0.08$.

Based on these preliminary observation, we suggest using our
newly-defined $k$-persistence metric as an experimental tool to guide
persistence-based load-balancing.
