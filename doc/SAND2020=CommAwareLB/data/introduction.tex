Consider the case where $\mathbf{P}=\{p_1;p_2\}$,
$\mathbf{O}=\{O_1;O_2\}$, and $load(O_1)=load(O_2)=\ell$ (in the sense
of time needed to perform these tasks), and denote
as follows the $4$ possible object/processor mappings:
$m_{ij}=\{O_1\to{p_i};O_2\to{p_j}\}$.
Both $m_{12}$ and $m_{21}$ have null imbalance, whereas for both
$m_{11}$ and $m_{22}$ $\mathcal{I}=1$.
In other words, in terms of object/processor balance, as has
been the scope of this entire report, of the $4$ possible
distributions, exactly $2$ are optimal and occur when exactly one
object in~$\mathbf{O}$ is assigned to each processor in~$\mathbf{P}$.

But now, consider the additional piece of information whereby an
some data must be communicated between $O_1$ and $O_2$ before the
work (with time $\ell$) can be completed by each.
For the sake of simplicity, assume and that it takes no time to
perform this data exchange when both objects reside on the same rank,
whereas it takes a time~$\tau>0$ to do it when they do not.
When factoring in this new parameter, we see that the total time $T$
that is required to perform the entire work (i.e., executing all
objects to completion) takes on the following values:
\[
T(m_{11}) = T(m_{22}) = 2\ell
\quad;\quad
T(m_{12}) = T(m_{21}) = \ell + \tau.
\]
Evidently, depending on the relative values of $\ell$ and $\tau$, the
best use of the available resources is not necessarily and optimal
object/processor distribution.
Specifically, these two concepts coincide in our example if, and only
if, $\tau<\ell$.

How the taking into account of the inter-object communication costs
will affect the overal compute time, and modify what is considered an
optimal distribution, is the focus of our future work.
