The goal of this chapter is to examine the statistical properties of
some cases for which they can be explicitly calculated, hereby
providing the basis for reference non-regression testing as well as
performance comparisons.

In particular, all observed load/processor distributions will be
compared to an ideal \emph{equi-distribution}, in which all
objects are assigned in such a way that all processors in the
collection~$\mathbf{P}$ have identical load.
We note that this does not require that all objects have the same
load.

Respectively denoting $\vert{L}\vert$ and $\delta_i$ the total load
across~$\mathbf{P}$, and the Dirac delta function on processor
$p_i\in\mathbf{P}$, we can denote
$\mathcal{L}=\frac{\vert{L}\vert}{\vert{\mathbf{P}}\vert}
\sum_{i=1}^{\vert{\mathbf{P}}\vert}\delta_i$ the
equi-distribution of loads (ignoring the object dispatch details)
on~$\mathbf{P}$, for which we have  the following statistics:
$\min{\mathcal{L}} = \max{\mathcal{L}}
= \overline{\mathcal{L}} = \frac{\vert{L}\vert{}}{\vert{\mathbf{P}}\vert}$,
$\sigma_{\mathcal{L}}^2 = \mu_{3,\mathcal{L}} = \mu_{4,\mathcal{L}} = 0$,
and
$\mathcal{I}_{\mathcal{L}}= 0$
for obvious reasons.

All observed load/processor distributions will have to be compared
against this ideal baseline.

