It is a natural question to wonder whether the optimality results
obtained in the idealized iso-load case may also be used as
approximations of the expected load-balancing results when the objects
are of relatively homogenous sizes.

\begin{table}[htb!]
\begin{center}
\begin{tabular}{lcccccc}
\hline
statistic & iteration
& $\varepsilon=0$
& $\varepsilon=10^{-3}$
& $\varepsilon=10^{-2}$
& $\varepsilon=10^{-1}$
& $\varepsilon=1$ \\
\hline\hline
\multirow{2}{*}{$\min{\mathcal{L}}$}
&$0$ &$22$ &$21.99$ &$22.03$ &$21.89$ &$22.14$ \\
&$4$ &$\bf39$ &$38.99$ &$38.92$ &$37.97$ &$37.65$ \\\hline
\multirow{2}{*}{$\max{\mathcal{L}}$}
&$0$ &$57$ &$57.00$ &$56.98$ &$57.22$ &$59.48$ \\
&$4$ &$\bf40$ &$40.00$ &$39.96$ &$40.20$ &$40.46$ \\\hline
\multirow{2}{*}{$\sigma_{\mathcal{L}}$}
&$0$ &$5.928$ &$5.928$ &$5.929$ &$5.96302$ &$6.921$ \\
&$4$ &$\bf0.2421$ &$0.2406$ &$0.2283$ &$0.4097$ &$0.4358$ \\\hline
\multirow{2}{*}{$\mathcal{I}_\mathcal{L}$}
&$0$ &$0.4592$ &$0.4591$ &$0.4588$ &$0.4647$ &$0.5211$ \\
&$4$ &$\bf0.024$ &$0.02405$ &$0.02301$ &$0.02889$ &$0.03470$ \\\hline
\end{tabular}
\end{center}
\caption{\label{t:iso-load-not-divisible-epsilon-i4}
Load/processor statistics before and after $4$ iterations of the
original and modified algorithms, with $4$ gossiping rounds, when
$\vert\mathbf{P}\vert$ does not divide $\vert\mathbf{O}\vert$, when
uniform random noise of increasing magnitude is added to the iso-load
case.}
\end{table}
For instance, \emph{ceteris paribus} as compared to the case
presentend in Table~\ref{t:iso-load-not-divisible}, but this time with 
$load(O_i)\sim\mathcal{U}(1-\varepsilon;1+\varepsilon)$, we obtain the
following results presented in in
Table~\ref{t:iso-load-not-divisible-epsilon-i4},with one experiment for
each of the reported values of~$\varepsilon below$ (now only
considering the algorithm using alternate 
criterion $\texttt{6'}$).

At first glance we may be tempted to hypothesize that the iso-load
approximations provide good approximations unless the object loads are
allowed to fluctuate within a few percentage points about the mean~$\ell$.
