In fact, we noticed that this finding can be explained by the fact
that the condition set forth in the algorithm for each overloaded
processor:
\[
\mathtt{4:} \quad \mathrm{\mathbf{while}} \; (L_i > (T \times L_{ave}))
\]
mandates that, after a variable number of iterations, each overloaded
processor no longer is, up to a certain relative threshold $T$.
This implies that, in the worst case, after completion of this loop,
\[
\label{eq:imbalance}
L_{\max} \le T \times L_{ave}
\quad \Longleftrightarrow \quad
\mathcal{I_D} < T - 1
\]
where $\mathcal{I}_D$ is the load imbalance of the distribution $D$ of
objects across the entire set of processors. This amounts to
saying that the objective function that the algorithm aims to minimize
is $F(D)=\mathcal{I_D}-T+1$, and that a sufficient stopping criterion
if $F(D)\ge0$ (we observe that $L_{ave}$ is by definition a constant
as no loss or gain of load may occur globally). 

However, $F(D)\ge0$ is by no means necessary, and it fact if it were,
there would be no guarantee that the algorithm would terminate in
finite time. This is however ensured by the fact that, given an
underloaded processor, the criterion of line \texttt{6} is tested for
all of its objects, of which there is only a finite number.
While this ensures termination of the while-loop in finite time --
this does not guarantee that \emph{any} transfer will have occurred at
all.

We now illustrate this problem with one representative test case
with $o=10^4$ objects initially distributed across $p=2^4$
amongst $n=2^{12}$ processors, of $i=10$ iterations of the original
Grapevine algorithm (via our \textsf{NodeGossiper} simulator), each of
each having $k=10$ gossiping rounds, with an overload threshold of
$T=1.0$ and a fanout factor of $f=6$.
We observe the following rejection rates caused by the criterion of line \texttt{6}:

\begin{center}
\begin{tabular}{lrrr}
\hline
iteration & transfers & rejected & rejection rate\\
(index)   & (number)  & (number) & (\%)\\
\hline\hline
 1 & 9084 & 154931 & 94.46\\
 2 &    4 &   1654 & 99.76\\
 3 &    1 &   1130 & 99.91\\
 4 &    7 &   2682 & 99.74\\
 5 &    6 &   2396 & 99.75\\
 6 &    2 &   1143 & 99.83\\
 7 &    1 &   1041 & 99.90\\
 8 &    0 &    882 & 100.0\\
 9 &    0 &    882 & 100.0\\
10 &    3 &   1405 & 99.79\\
\hline
\end{tabular}
\end{center}
These immediately hint at the fact that the decision criterion is too
tight. And indeed, it enforces strict monotony for each of the
underloaded processors; in other words, it uses the ``taxicab'' norm
($\vert\vert\cdot\vert\vert_1$) to minimize in the
$\vert\vert\cdot\vert\vert_\infty$ sense: this criterion is therefore
not adapted to considered minimization problem.
Another way to look at the problem is to see it as an attempt to
decrease the load of underloaded processors while never allowing a
single underloaded one to become overloaded -- even when this may
improve the global imbalance.

As a result, these almost full rejection rates, limits load-balancing
to a noticeable decrease of $\mathcal{I}$ during the
first iteration of the algorithm, after which esssentially no further
improvements occur while $\mathcal{I}$ remains stuck in a local
minimum. Furthermore, increasing $k$ and allowing for higher values of
$T$ does not substantially affect the outcome on average (with the
exception of the occasional convergence due to a nice original layout).
This is illustrated in
Figure~\ref{Grapevine-n4096-p16-o10000-uniform-i10-k10-f6-t1_0}.




