We now study the results of the modified algorithm, with alternate
criterion \texttt{6'}, when used in the \textsf{NodeGossiper}
simulator for the same case as above.

The new acceptance/rejection results are as follows:
\begin{center}
\begin{tabular}{@{}lrrr@{}}
\hline
iteration & transfers & rejected & rejection rate\\
(index)   & (number)  & (number) & (\%)\\
\hline\hline
 1 & 11292 &  648 & 5.427\\
 2 &  4044 & 3603 & 47.12\\
 3 &  2201 & 3412 & 60.79\\
 4 &  1324 & 3586 & 73.03\\
 5 &   765 & 3171 & 80.56\\
 6 &   410 & 2969 & 87.87\\
 7 &   247 & 2794 & 91.88\\
 8 &   159 & 2749 & 94.53\\
 9 &   120 & 2682 & 95.72\\
10 &    72 & 2643 & 97.35\\
\hline
\end{tabular}
\end{center}

In contrast to what was happening with the original criterion
\texttt{6}, we see now that the rejection rate is almost null
initially, then slowly increases as the global imbalance rapidly decreases.
In fact, with already more than acceptable values of $\mathcal{I}$,
additional iterations continue to improve the outcome, hereby
experimentally validating the preceding theoretical results. This is
examplified by comparing the values of $\mathcal{I}$ in both cases:
\begin{center}
\begin{tabular}{@{}lrr@{}}
\hline
iteration & criterion \texttt{6}  & criterion \texttt{6'} \\
(index)   & ($\mathcal{I}$) & ($\mathcal{I}$) \\
\hline\hline
 0 & 280 &   280\\
 1 & 187 &  3.34\\
 2 & 187 &  1.60\\
 3 & 187 & 0.873\\
 4 & 185 & 0.632\\
 5 & 183 & 0.632 \\
 6 & 183 & 0.626 \\
 7 & 183 & 0.626 \\
 8 & 183 & 0.626 \\
 9 & 182 & 0.626 \\
10 & 182 & 0.623 \\
\hline
\end{tabular}
\end{center}
We note, in particular, that the modified algorithm has not fully run
its course after iteration $10$, and continues to improve
$\mathcal{I}$, albeit modestly, while the original algorithm has
essentially converged to a very sub-optimal local minimum and is no
longer able to improve the overall imbalance after a few steps.
