The experiment we are presenting used an object mapping read from the
\texttt{.vom} files, whose aggregrated per-processor times are
illustrated in Figure~\ref{vt-example_rank_loads}: a cursory visual inspection
of this histogram readily indicates a poorly-balanced initial distribution.

\begin{table}[htb!]
\begin{center}
\begin{tabular}{@{}lrrrrrrrrr@{}}
\hline
$\mathcal{D}$ &
$\vert\mathcal{D}\vert$ &
$\min{\mathcal{D}}$ & 
$\overline{\mathcal{D}}$ & 
$\max{\mathcal{D}}$ &
$R_{\mathcal{D}}$ &
$\sigma_{\mathcal{D}}$ &
$\gamma_{1,\mathcal{D}}$ &
$\gamma_{2,\mathcal{D}}$ &
$\mathcal{I}_{\mathcal{D}}$ \\
\hline\hline
\textbf{O} &
$100$     & $0.00026703$ & $0.042076$ & $0.30404$ &
$0.30377$ & $0.074587$   &$2.8855$    & $9.8574$ &
N/A \\\hline
$\mathcal{L}$ &
$8$      & $0.004673$ & $0.52595$ & $2.3209$ &
$2.3162$ & $0.69632$  & $2.030$   & $5.5987$ &
$3.4128$\\\hline
\end{tabular}
\end{center}
\caption{\label{t:vt-example} Object mapping statistics of the initial
object/processor load distribution of a \textsf{VT}-based simulation
on~8 ranks.}
\end{table}

In order to provide a quantitative assessment of the distributions
obtained after having applied the \textsf{NodeGossiper} (with either
of its two of variants), with respect to the initial situation, some
key statistical properties are shown in Table~\ref{t:vt-example}: the 
first row provides the values for the object times themselves, while
the second line presents those for our variable of interest, the
per-processor load~$\mathcal{L}$. 
These statistics indicate in particular overall load imbalance
that is almost half the size of the rank set, with the most heavily
loaded rank being assigned $500$ times more work that the least loaded
one.
We shall now examine how both versions of the algorithm perform in
this case.

  
