In order to ensure a faster convergence of the algorithm, we propose
the following adapted criterion:
\par\textbf{Lemma~1:}\\
The following alternate criterion:
\[
\mathtt{6':} \qquad \mathrm{\mathbf{if}} \; load(O_i) < L_i - L_X
\; \mathrm{\mathbf{then}}
\]
ensures that the objective function $F$ monotonically decreases.
\par\textit{Proof:}\\
Consider a strictly overloaded processor $p_i$ and a strictly
underloaded one $p_X$ in an processor/object distribution $D$, with
respective loads $L_i$ and $L_X$.
Consider also an object $O_i\in{p_i}$ such that $\ell=load(O_i) <
L_i-L_X$ (both sides being necessarily positive by hypothesis on
loads); we can thus distinguish the two following disjoint cases:

\begin{enumerate}
\item
If $\ell \le \frac{1}{2}(L_i-L_X)$, then $L_i-\ell\ge{}L_X+\ell$ and
thus $L_X+\ell\le L_i-\ell < L_i$.
\item
If $\ell > \frac{1}{2}(L_i-L_X)$, then $L_i-\ell<L_X+\ell$ and
thus $L_i-\ell < L_X+\ell < L_i \Leftrightarrow \ell < L_i - L_X.$
\end{enumerate}
Therefore, we can assert that, overall,
$\max{(L_i-\ell, L_X+\ell)} < L_i \Leftrightarrow \ell < L_i - L_X$.
Now, recall that
\[
F(D) = \frac{L_{\max}}{L_{ave}} - T
\ge \frac{L_i}{L_{ave}} - T,
\]
with equality if and only if $p_i$ is such that $L_i=L_{\max}$ in $D$
(i.e., it has maximum load in the original distribution).
Therefore, $\ell<L_i-L_X$ ensures that
\begin{equation}
\label{eq:imbalance-change}
\frac{\max{(L_i-\ell, L_X+\ell)}}{L_{ave}} -T
< \frac{L_i}{L_{ave}} - T \le F(D)
\end{equation}
Finally, when considering the new load/processor distribution $D'$ that
results from the transfer of $O_i$ from $p_i$ to $p_X$, and denoting
$L_i'=L_i-\ell$ and $L_X'=L_X+\ell$ their respective new loads, we are faced
with a trichotomy of possible cases (the two first ones are not necessarily disjoint):
\begin{enumerate}
\item
If $p_i$ is such that $L_i'=L_{\max}'$ in $D'$ (i.e., $p_i$ has
maximum load in the new distribution $D'$), then
$\max{(L_i-\ell,L_X+\ell)}=L_i-\ell=L_i'$ and
(\ref{eq:imbalance-change}) yields:
\[
F(D') = \frac{L_{\max}'}{L_{ave}} - T
 = \frac{L_i'}{L_{ave}} - T = \frac{L_i -\ell}{L_{ave}} - T
< \frac{L_i}{L_{ave}} - T \le F(D)
\]
\item
If $p_X$ is such that $L_X'=L_{\max}'$ in $D'$ (i.e., $p_X$ has
maximum load in the new distribution $D'$), then
$\max{(L_i-\ell,L_X+\ell)}=L_X+\ell=L_X'$ and
(\ref{eq:imbalance-change}) yields:
\[
F(D') = \frac{L_{\max}'}{L_{ave}} - T
 = \frac{L_X'}{L_{ave}} - T  = \frac{L_X+\ell}{L_{ave}} - T
< \frac{L_i}{L_{ave}} - T \le F(D)
\]
\item
If a processor $p_Y\not\in\{p_i,p_X\}$ is such that $L_Y'=L_{\max}'$
in $D'$ (i.e., neither $p_i$ nor $p_X$ have maximum load in the new
distribution $D'$), then necessarily $L_Y'=L_Y$ because the transfer did not
affect $p_Y$, and thus necessarily $L_Y'\le{}L_i$ and
(\ref{eq:imbalance-change}) yields:
\[
F(D') = \frac{L_{\max}'}{L_{ave}} - T
 = \frac{L_Y'}{L_{ave}} - T  = \frac{L_Y}{L_{ave}} - T
\le \frac{L_i}{L_{ave}} - T \le F(D).
\]
Furthermore, in this case,
$\frac{L_Y}{L_{ave}}-T\le\frac{L_i}{L_{ave}}-T$ is \emph{not} strict
if and only if $P_Y$ also had maximum load in $D$; because there is
only a finite number of such maximally-overloaded processors, we are
guaranteed that in a finite number of iterations the inequality will
become strict.
\end{enumerate}
We can therefore conclude that, overall,
\[
\ell < L_i - L_X \Longrightarrow F(D') < F(D).
\]
\hfill\qed\\

We remark that the new, modified criterion can be equivalently written
as
\[
\mathtt{6':} \qquad \mathrm{\mathbf{if}} \; L_X + load(O_i) < L_i
\; \mathrm{\mathbf{then}}
\]
which is indeed less strict than the the original one, for it allows,
in particular, one underloaded processor to land in overloaded
territory after a transfer. However, what is ensured is the maximum
norm will not increase.
In addition, Lemma~1 ensures that, as long as one can find at least
one object satisfying this criterion on at least one overloaded
processor, then the optimization can continue. However, once it is no
longer possible to find such a combination, then $F$ may no longer
decrease: this new criterion thus also provides a stopping criterion.

While this criterion will provide more opportunities for overload
transfers than the original one, one may wonder whether it could not
be further relaxed, hereby allowing for even lower rejection rates.
This question is quicly answered by the following, with the same
notations a in Lemma~1:
\par\textbf{Lemma~2:}\\
If $p_i$ is a processor with maximum load in $D$, and
\[
(\exists O_i \in p_i)\; (\exists p_X \in D) \quad \ell = load(O_i) \ge{} L_i - L_X
\]
then if $O_i$ is transferred from $p_X$ to $p_i$, the objective
function $F$ does not decrease (and possibly increases).
\par\textit{Proof:}\\
If $p_i$ has maximum load in the object/load distribution $D$, and one
can find $O_i\in{}p_i$ and $p_X\in{}D$, then by definition of $F$ one
has, on one hand:
\[
\frac{L_X + \ell}{L_{ave}} - T \ge \frac{L_i}{L_{ave}} - T = F(D).
\]
On the other hand, in the new distribution $D'$ obtained by
transferring $O_i$ from $p_i$ to $p_X$, one has:
\[
\frac{L_X + \ell}{L_{ave}} - T = \frac{L_X'}{L_{ave}}
\le \frac{\max{L_{\cdot}'}}{L_{ave}} - T = F(D').
\]
Combining the two above inequalities thus yields
$F(D')\ge{}F(D)$. Furthermore, if $(O_i,p_x)$ is such that $L_X'$ is
not a maximally-overloaded processor in the new distribution, then the
latter inequality is strict, in which case $F$ increases.
\hfill\qed.\\
As a result of Lemma~1 and Lemma~2, we can now assert the following:
\par\textbf{Proposition [Optimal Load-Transfer Criterion]:}\\
The following alternate criterion:
\[
\mathtt{6':} \qquad \mathrm{\mathbf{if}} \; load(O_i) < L_i - L_X
\; \mathrm{\mathbf{then}}
\]
is optimal for the load transfer strategy of Algorithm~2.
\par\textit{Proof:}\\
From Lemma~1 we know that this alternate criterion ensures monotonic
is sufficient to ensure that $F$ monotonically decrease.

Furthermore, from Lemma~2 we know that is this criterion is not
verified for at least one particular case, then $F$ will no longer
monotonically decrease (and will possibly increase if $(O_i,p_X)$ is
such that $L_X'$ is not maximal in $D'$.

Therefore, alternate criterion \texttt{6'}, being necessary and
sufficient, is optimal for the considered optimization strategy.
\hfill\qed.\\
