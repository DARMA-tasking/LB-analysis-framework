In fact, we noticed that this finding can be explained by the fact
that the condition set forth in the algorithm for each overloaded
processor:
\[
\mathtt{4:} \quad \mathrm{\mathbf{while}} \; (L_i > (T \times L_{avg}))
\]
mandates that, after a variable number of iterations, each overloaded
processor no longer is, up to a certain relative threshold $T$.
This implies that, in the worst case, after completion of this loop,
\begin{equation}
\label{eq:imbalance}
L_{\max} \le T \times L_{avg}
\quad \Longleftrightarrow \quad
\mathcal{I_P} < T - 1
\end{equation}
where $\mathcal{I}_D$ is the load imbalance of the distribution $D$ of
objects across the entire set of processors. This amounts to
saying that the objective function 
that the algorithm tries to minimize is $F(P)=\mathcal{I_P}-T+1$ and
stops when it reaches 0 (we observe that $L_{avg}$ is by definition a
constant as no loss or gain of load may occur globally).

The criterion in line \texttt{6} of the original algorithm does indeed
guarantee that $L_{\max}$ can only decrease after a load transfer has
occurred, as no underloaded processor may become overloaded. As a
result, the new value of $F$ resulting from a transfer can only be
smaller or equal to the preceding one.
However, there is an important \emph{caveat} here, which is that all
objects $O_i$ belonging to $p_i$ are tried until their (finite) list
exhausted, ensuring termination of the while-loop in finite time --
but this does not guarantee that \emph{any} transfer will have
occurred. In order to ensure swift convergence of the algorithm, we
therefore ought to minimize the \emph{rejection} rate of line
\texttt{6}. 

an overload transfer,  of the original algorithm is that,
while it guarantees decreasing overloads while preventing 

has only one variable ($L_{\max}$), i.e., the maximum norm of the
loads, while the decision criterion enforces strict monotonicity for
each underloaded processors (this should be compatible with a
taxicab norm on the subset of underloaded processors).

We firstly establish the following result
\par\textbf{Proposition:}\\
The following criterion:
\[
\mathtt{6:} \qquad \mathrm{\mathbf{if}} \; load(O_i) < L_i - L_X
\; \mathrm{\mathbf{then}}
\]
is optimal for the considered objective function when $T=1$.
\par\textit{Proof:}\\
Consider a strictly overloaded processor $p_i$ and a strictly
underloaded one $p_X$ in a set of processors $P$, whose \emph{local
imbalance} 

with global imbalance
$\mathcal{I}_P$, with respective loads $L_i$ and $L_X$.
By definition, 
\[
\frac{L_X}{L_{ave}} - 1 < \frac{L_i}{L_{ave}} - 1 \le \mathcal{I}
\]
with equality of the second inequality if and only if $P_i$ belongs to
the subset of the most overloaded processors. Therefore, the local
imbalance of these two processors is given by

Consider now an objet $O\in P_i$, with contribution $l_O$ to $L_i$,
which is transferred from $P_i$ to $P_X$. Denoting $\mathcal{I}_k$ the
 of any given processor $P_k$, 
enforce the following monotonicity condition:
\[
\mathcal{I}' \le \mathcal{I}
\]
which is equivalent to
\[
\mathcal{I}' \le \mathcal{I}
\]
