By \emph{Dirac distribution}, we mean here any case where all objects
are assigned to a single processor in the non-empty
collection~$\mathbf{P}$. Without any loss of generality for the sake
of statistical properties, we can assume that this processor is $p_1$,
and we therefore consider the load distribution
$\mathcal{L}=\vert{L}\vert\delta_1$.
Also denoting $\vert\mathbf{P}\vert$ the cardinality of ~$\mathbf{P}$, we
obviously have the following statistics for $\mathcal{L}$:
\begin{align*}
\min{\mathcal{L}}
&= 0,\\
\max{\mathcal{L}}
&= \vert{L}\vert{},\\
\overline{\mathcal{L}}
&= \frac{\vert{L}\vert}{{\vert\mathbf{P}\vert}}.
\end{align*}
Therefore, using the latter equation, we obtain the variance of
$\mathcal{L}$ as:
\begin{align*}
\sigma_{\mathcal{L}}^2
&= \frac{1}{{\vert\mathbf{P}\vert}}\sum_{i=1}^{\vert\mathbf{P}\vert}\left(
\vert{L}\vert\delta_1(i) - \frac{\vert{L}\vert}{{\vert\mathbf{P}\vert}}\right)^2\\
&= \frac{\vert{L}\vert^2}{{\vert\mathbf{P}\vert}}\left[\left(
1 - \frac{1}{{\vert\mathbf{P}\vert}}\right)^2 +\frac{{\vert\mathbf{P}\vert}-1}{{\vert\mathbf{P}\vert}^2}\right]\\
&= \frac{\vert{L}\vert^2({\vert\mathbf{P}\vert} - 1)}{{\vert\mathbf{P}\vert}^2}
\underset{{\vert\mathbf{P}\vert}\to\infty}{\sim}\frac{\vert{L}\vert^2}{{\vert\mathbf{P}\vert}}.
\end{align*}
Concerning the third and fourth central moments of~$\mathcal{L}$, we
have on one hand:
\begin{align*}
\mu_{3,\mathcal{L}}
&= \frac{1}{{\vert\mathbf{P}\vert}}\sum_{i=1}^{\vert\mathbf{P}\vert}\left(
\vert{L}\vert\delta_1(i) - \frac{\vert{L}\vert}{{\vert\mathbf{P}\vert}}\right)^3\\
&= \frac{\vert{L}\vert^3}{{\vert\mathbf{P}\vert}}\left[\left(
1 - \frac{1}{{\vert\mathbf{P}\vert}}\right)^3 - \frac{{\vert\mathbf{P}\vert}-1}{{\vert\mathbf{P}\vert}^3}\right]\\
&= \frac{\vert{L}\vert^3}{{\vert\mathbf{P}\vert}}\times\frac{({\vert\mathbf{P}\vert}-1)^3 - {\vert\mathbf{P}\vert} + 1}{{\vert\mathbf{P}\vert}^3}\\
&= \frac{\vert{L}\vert^3\left({\vert\mathbf{P}\vert}^2 - 3{\vert\mathbf{P}\vert} + 2\right)}{{\vert\mathbf{P}\vert}^3},
\end{align*}
and on the other hand:
\begin{align*}
\mu_{4,\mathcal{L}}
&= \frac{1}{{\vert\mathbf{P}\vert}}\sum_{i=1}^{\vert\mathbf{P}\vert}\left(
\vert{L}\vert\delta_1(i) - \frac{\vert{L}\vert}{{\vert\mathbf{P}\vert}}\right)^4\\
&= \frac{\vert{L}\vert^4}{{\vert\mathbf{P}\vert}}\left[\left(
1 - \frac{1}{{\vert\mathbf{P}\vert}}\right)^4 +\frac{{\vert\mathbf{P}\vert}-1}{{\vert\mathbf{P}\vert}^4}\right]\\
&= \frac{\vert{L}\vert^4}{{\vert\mathbf{P}\vert}}\times\frac{({\vert\mathbf{P}\vert} - 1)^4 + {\vert\mathbf{P}\vert} - 1}{{\vert\mathbf{P}\vert}^4}\\
&= \frac{\vert{L}\vert^4\left({\vert\mathbf{P}\vert}^3 -
4{\vert\mathbf{P}\vert}^2 + 6{\vert\mathbf{P}\vert} - 3\right)}
{{\vert\mathbf{P}\vert}^4}.
\end{align*}
The above results allow us to compute the skewness and kurtosis of the
distribution, when ${\vert\mathbf{P}\vert}>1$, respectively as follows:
\[
\gamma_{1,\mathcal{L}}
= \frac{\mu_{3,\mathcal{L}}}{\sigma_{\mathcal{L}}^3}\\
= \frac{{\vert\mathbf{P}\vert}^2 - 3{\vert\mathbf{P}\vert} + 2}{({\vert\mathbf{P}\vert}-1)^{\sfrac{3}{2}}}
\underset{{\vert\mathbf{P}\vert}\to\infty}{\sim}\sqrt{{\vert\mathbf{P}\vert}},
\]
and
\[
\gamma_{2,\mathcal{L}}
= \frac{\mu_{4,\mathcal{L}}}{\sigma_{\mathcal{L}}^4}\\
= \frac{{\vert\mathbf{P}\vert}^3 - 4{\vert\mathbf{P}\vert}^2 + 6{\vert\mathbf{P}\vert} -3}{({\vert\mathbf{P}\vert} - 1)^2}
\underset{{\vert\mathbf{P}\vert}\to\infty}{\sim}{\vert\mathbf{P}\vert}.
\]
Finally, we observe that
$\mathcal{I}_{\mathcal{L}}= {\vert\mathbf{P}\vert}-1\underset{{\vert\mathbf{P}\vert}\to\infty}{\sim}{\vert\mathbf{P}\vert}$.

The results above shall be used for statistical verification of
idealized cases, as upper bounds on worst-case scenarii (paying
attention to the fact that most statistical packages report
\emph{kurtosis excess}, i.e., $\gamma_2-3$).

