By \emph{Dirac distribution}, we mean here the case where all objects
are assigned to a single processor in the non-empty
collection~$\mathbf{P}$. Without any loss of generality for the sake
of statistical properties, we can assume that this processor is $p_1$,
and we therefore consider the load distribution
$\mathcal{L}=\vert{L}\vert{}\delta_1$, where $\vert{L}\vert$ and $\delta_1$
respectively denote the total load and the Dirac delta function on
$p_1$.
Also denoting $\vert{N}\vert$ the cardinality of ~$\mathbf{P}$, we
obviously have the following statistics for $\mathcal{L}$:
\begin{align*}
\min{\mathcal{L}}
&= 0,\\
\max{\mathcal{L}}
&= \vert{L}\vert{},\\
\overline{\mathcal{L}}
&= \frac{\vert{L}\vert{}}{N}.
\end{align*}
Therefore, using the latter equation, we obtain the variance of
$\mathcal{L}$ as:
\begin{align*}
\sigma_{\mathcal{L}}^2
&= \frac{1}{N}\sum_{i=1}^n\left(
\vert{L}\vert{}\delta_1 - \frac{\vert{L}\vert}{N}\right)^2\\
&= \frac{\vert{L}\vert^2}{N}\left[\left(
1 - \frac{1}{N}\right)^2 +\frac{N-1}{N^2}\right]\\
&= \frac{\vert{L}\vert^2(N - 1)}{N^2}
\underset{N\to\infty}{\sim}\frac{\vert{L}\vert{}^2}{N}.
\end{align*}
Concerning the third and fourth central moments of~$\mathcal{L}$, we
have on one hand:
\begin{align*}
\mu_{3,\mathcal{L}}
&= \frac{1}{N}\sum_{i=1}^n\left(
\vert{L}\vert{}\delta_1 - \frac{\vert{L}\vert}{N}\right)^3\\
&= \frac{\vert{L}\vert^3}{N}\left[\left(
1 - \frac{1}{N}\right)^3 - \frac{N-1}{N^3}\right]\\
&= \frac{\vert{L}\vert^3}{N}\times\frac{(N-1)^3 - N + 1}{N^3}\\
&= \frac{\vert{L}\vert{}^3(N^2 - 3N + 2)}{N^3},
\end{align*}
and on the other hand:
\begin{align*}
\mu_{4,\mathcal{L}}
&= \frac{1}{N}\sum_{i=1}^n\left(
\vert{L}\vert{}\delta_1 - \frac{\vert{L}\vert}{N}\right)^4\\
&= \frac{\vert{L}\vert^4}{N}\left[\left(
1 - \frac{1}{N}\right)^4 +\frac{N-1}{N^4}\right]\\
&= \frac{\vert{L}\vert^4}{N}\times\frac{(N - 1)^4 + N - 1}{N^4}\\
&= \frac{\vert{L}\vert{}^4(N^3 - 4N^2 + 6N -3)}{N^4}.
\end{align*}
The above results allow us to compute the skewness and kurtosis of the
distribution, when $N>1$, respectively as follows:
\[
\gamma_{1,\mathcal{L}}
= \frac{\mu_{3,\mathcal{L}}}{\sigma_{\mathcal{L}}^3}\\
= \frac{N^2 - 3N + 2}{(N-1)^{\sfrac{3}{2}}}
\underset{N\to\infty}{\sim}\sqrt{N},
\]
and
\[
\gamma_{2,\mathcal{L}}
= \frac{\mu_{4,\mathcal{L}}}{\sigma_{\mathcal{L}}^4}\\
= \frac{N^3 - 4N^2 + 6N -3}{(N - 1)^2}
\underset{N\to\infty}{\sim}N.
\]
Finally, we observe that 
$\mathcal{I}_{\mathcal{L}}= N-1\underset{N\to\infty}{\sim}N$.

The results above shall be used for statistical verification of
idealized cases, as upper bounds on worst-case scenarii (paying
attention to the fact that most statistical packages report
\emph{kurtosis excess}, i.e., $\gamma_2-3$).

