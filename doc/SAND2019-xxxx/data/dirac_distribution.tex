By \emph{Dirac distribution}, we mean here the case where all objects
are assigned to a single processor in the
collection~$\mathbf{P}$. Without any loss of generality for the sake
of statistical properties, we can assume that this processor is $p_1$,
and we therefore consider the load distribution
$\mathcal{L}=\vert{L}\vert{}\delta_1$, where $\vert{L}\vert$ and $\delta_1$
respectively denote the total load and the Dirac delta function on
$p_1$.
Also denoting $\vert{N}\vert$ the cardinality of ~$\mathbf{P}$, we
obviously have the following statistics for $\mathcal{L}$:
\begin{align*}
\min{\mathcal{L}}
&= \vert{L}\vert{} \;\mathrm{if}\;N=0;\;\vert{L}\vert{}\;\mathrm{otherwise},\\
\max{\mathcal{L}}
&= \vert{L}\vert{},\\
\overline{\mathcal{L}}
&= \frac{\vert{L}\vert{}}{N}.
\end{align*}
Therefore, using the latter equation, we obtain the variance of
$\mathcal{L}$ as:
\begin{align*}
\sigma_{\mathcal{L}}^2
&= \frac{1}{N}\sum_{i=1}^n\left(
\vert{L}\vert{}\delta_1 - \frac{1}{N}\right)^2\\
&= \frac{1}{N}\left[\left(
\vert{L}\vert{} - \frac{1}{N}\right)^2 +\frac{N-1}{N^2}\right]\\
&= \frac{\vert{L}\vert{}^2N-1}{N^2}
\underset{N\to\infty}{\sim}\frac{\vert{L}\vert{}^2}{N}.
\end{align*}
Concerning the third and fourth central moments of~$\mathcal{L}$, we
have on one hand:
\begin{align*}
\mu_{3,\mathcal{L}}
&= \frac{1}{N}\sum_{i=1}^n\left(
\vert{L}\vert{}\delta_1 - \frac{1}{N}\right)^3\\
&= \frac{1}{N}\left[\left(
\vert{L}\vert{} - \frac{1}{N}\right)^3 - \frac{N-1}{N^3}\right]\\
&= \frac{(\vert{L}\vert{}N-1)^3 - N + 1}{N^4}\\
&= \frac{\vert{L}\vert{}^3N^2 - 3\vert{L}\vert{}^2N
+ 3\vert{L}\vert{}- 1}{N^3},\\
\end{align*}
and on the other hand:
\begin{align*}
\mu_{4,\mathcal{L}}
&= \frac{1}{N}\sum_{i=1}^n\left(
\vert{L}\vert{}\delta_1 - \frac{1}{N}\right)^4\\
&= \frac{1}{N}\left[\left(
\vert{L}\vert{} - \frac{1}{N}\right)^4 +\frac{N-1}{N^4}\right]\\
&= \frac{(\vert{L}\vert{}N-1)^4 + N - 1}{N^5}\\
&= \frac{\vert{L}\vert{}^4N^3 - 4\vert{L}\vert{}^3N^3
+ 6\vert{L}\vert{}^2N - 4\vert{L}\vert{} + 1}{N^4}.\\
\end{align*}
The above results finally allow us to compute the skewness and
kurtosis of the distribution, respectively as follows:
\begin{align*}
\gamma_{1,\mathcal{L}}
&= \frac{\mu_{3,\mathcal{L}}}{\sigma_{\mathcal{L}}^3}\\
&= \frac{\vert{L}\vert{}^3N^2-3\vert{L}\vert{}^2N+3\vert{L}\vert{}-1}
{(\vert{L}\vert{}^2N-1)^{\sfrac{3}{2}}}
\underset{N\to\infty}{\sim}\sqrt{N},
\end{align*}
and
\begin{align*}
\gamma_{2,\mathcal{L}}
&= \frac{\mu_{4,\mathcal{L}}}{\sigma_{\mathcal{L}}^4}\\
&= \frac{\vert{L}\vert{}^4N^3 - 4\vert{L}\vert{}^3N^3
+ 6\vert{L}\vert{}^2N - 4\vert{L}\vert{} + 1}
{(\vert{L}\vert{}^2N-1)^2}
\underset{N\to\infty}{\sim}N.
\end{align*}
Finally, we can also observe that
\[
\mathcal{I}_{\mathcal{L}}
= 0 \;\mathrm{if}\;N=0;\;N-1\underset{N\to\infty}{\sim}N\;\mathrm{otherwise}.
\]
The results above shall be used for statistical verification of
idealized cases, as upper bounds on worst-case scenarii (paying
attention to the fact that most statistical packages report
\emph{kurtosis excess}, i.e., $\gamma_2-3$).

