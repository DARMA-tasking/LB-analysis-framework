
We compared the results obtained with the two different
implementations (Grapevine algorithm with original criterion and
\texttt{6}, vs. NodeGossiper with modified criterion \texttt{6'}) by
using in both cases the following input parameters: $i=5$, $k=f=4$,
and an overload threshold $t=1_0$, and identical seedings of the
pseudo-random number generator used by the samplers in the two phases
of the algorithm.

\begin{table}[htb!]
\begin{center}
\begin{tabular}{@{}lrrrrrrrr@{}}
\hline
criterion &
$\min{\mathcal{L}}$ & 
$\max{\mathcal{L}}$ &
$R_{\mathcal{L}}$ &
$\sigma_{\mathcal{L}}$ &
$\gamma_{1,\mathcal{L}}$ &
$\gamma_{2,\mathcal{L}}$ &
$\mathcal{I}_{\mathcal{L}}$ \\
\hline\hline
\texttt{6} &
$0.28542$ & $1.7508$ & $1.4654$ &
$0.46454$ & $2.2364$ & $6.0655$ &
$2.3288$  \\\hline
\texttt{6'} &
$0.49539$  & $0.57315$ & $0.077759$ &
$0.024726$ & $0.77394$ & $2.3940$ &
$0.089739$ \\\hline
\end{tabular}
\end{center}
\caption{\label{t:comparative_results} Object mapping statistics
obtained after $5$ iterations of the load-balancing algorithm with
both version of the decision criteria (\texttt{6} vs. \texttt{6'})
when $k=f=4$.}
\end{table}

The results obtained for each experiment  presented in
Table~\ref{t:comparative_results}. Cardinalities and averages are
omitted as they are identical to those indicated in the last row of
Table~\ref{t:vt-example}, as both are invariant because we do not
allow the number of ranks (nor of course the total load) to change
during the load-balancing process.
This comparison speaks for itself, and we confirm with this real case
what we had established from first principles and analytical cases.

 
