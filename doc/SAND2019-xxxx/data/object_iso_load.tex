To start, we place ourselves in the idealize case where all objects
$O_i\in\mathbf{P}$ have the same load $load(O_i)=\ell$.
Although it is safe to assume that this ideal case will very seldom,
if ever, be encountered in any real case, it is important that any
proposed load-balancing algorithm be able to perform optimally in this
case.

Evidently, if $\vert\mathbf{P}\vert$ divides $\vert\mathbf{O}\vert$,
then any  equi-distribution is optimal and is obtained by assigning
exactly $\tfrac{\vert\mathbf{O}\vert}{\vert\mathbf{P}\vert}$ object(s)
to each processor in $\mathbf{P}$.
We thus present below the respective results of the
Grapevine algorithm with both original criterion \texttt{6} (left) and
our modified criterion \texttt{6'}, for a case where
$\mathbf{P}=10^2$, $\mathbf{O}=10^4$, $i=k=f=4$, $t=1.0$, and where
the unit load is chosen to be $\ell=1$:
\begin{center}
\begin{tabular}{lcccc}
\hline
statistic & \textbf{optimal} & iteration
& criterion \texttt{6} & criterion \texttt{6'} \\
\hline\hline
\multirow{2}{*}{$\min{\mathcal{L}}$}
&\multirow{2}{*}{$\bf100$}
&$0$ &\multicolumn{2}{c}{$80$} \\
&&$4$ &$99$    &$\bf100$ \\\hline
\multirow{2}{*}{$\max{\mathcal{L}}$}
&\multirow{2}{*}{$\bf100$}
&$0$ &\multicolumn{2}{c}{$123$} \\
&&$4$ &$116$   &$\bf100$ \\\hline
\multirow{2}{*}{$\sigma_{\mathcal{L}}$}
&\multirow{2}{*}{$\bf0.00$}
&$0$ &\multicolumn{2}{c}{$9.18$} \\
&&$4$ &$2.34$  &$\bf0.00$ \\\hline
\multirow{2}{*}{$\mathcal{I}_\mathcal{L}$}
&\multirow{2}{*}{$\bf0.00$}
&$0$ &\multicolumn{2}{c}{$0.23$} \\
&&$4$ &$0.16$  &$\bf0.00$ \\\hline
\end{tabular}
\end{center}
The above results demonstrate that even with small numbers of fanout,
gossiping round, and iterations, our alternatre approach converges to
the optimal solution whereas the original one does not, even with this
simple academic case. This first baseline case is thus discriminating
enough to be retained.

In order to extend the set of baseline cases, we now consider a
slightly broader class of distributions, where all 
objects still have equal load $\ell$, but where it is no longer
required that $\vert\mathbf{P}\vert$ divide $\vert\mathbf{O}\vert$.
In this context, denoting
$q=\lfloor\sfrac{\vert\mathbf{O}\vert}{\vert\mathbf{P}\vert}\rfloor$
and $r=\vert\mathbf{O}\vert\bmod\vert\mathbf{P}\vert$, consider a
distribution of $q$ objects on $\vert\mathbf{P}\vert-r$ processors 
($0\le{r}<\vert\mathbf{P}\vert$ by definition of the Euclidean
division) and $q+1$ objects on the remaining $r$
processors. All objects are indeed assigned because
$r(q+1)+(\vert\mathbf{P}\vert-r)q=r+q\vert\mathbf{P}\vert=\vert\mathbf{O}\vert$.
We thus have, if $r\neq0$ (i.e., in the non-divisible case):
\begin{align*}
\min{\mathcal{L}}
&= q\ell,\\
\max{\mathcal{L}}
&= (q+1)\ell,\\
\overline{\mathcal{L}}
&= \frac{\vert\mathbf{O}\vert}{\vert\mathbf{P}\vert}\ell,
\end{align*}
wherefrom we find that
\[
\mathcal{I}_{\mathcal{L}} =
\frac{q+1}
{\sfrac{\vert\mathbf{O}\vert}{\vert\mathbf{P}\vert}} - 1
=
\frac{q\vert\mathbf{P}\vert + \vert\mathbf{P}\vert
- \vert\mathbf{O}\vert}{\vert\mathbf{O}\vert}
=
\frac{\vert\mathbf{P}\vert - r}{\vert\mathbf{O}\vert}
> 0.
\]
We note that the case where $r=0$, we have instead
\[
\mathcal{I}_{\mathcal{L}} =
\frac{q}
{\sfrac{\vert\mathbf{O}\vert}{\vert\mathbf{P}\vert}} - 1
=
\frac{q\vert\mathbf{P}\vert - \vert\mathbf{O}\vert}
{\vert\mathbf{O}\vert}
= 0,
\]
which amounts to the previously discussed distribution where
$\vert\mathbf{P}\vert$ divides $\vert\mathbf{O}\vert$, that we already
proved to be optimal.

Furthermore, as soon as $r\neq0$, any other distribution assigning
more than $q+1$ objects to at least one processor has
a larger imbalance, because the denominator
$\overline{\mathcal{L}}$ in $\mathcal{I}_{\mathcal{L}}$ is fixed.
Therefore, a distribution $\mathcal{L}$ of
$\vert\mathbf{O}\vert$ objects with identical load $\ell$ across
$\vert\mathbf{P}\vert$ processors is optimal if and only if it can be
written (with no loss of generality, up to a re-indexing of
processors) as follows:
\[
\mathcal{L}=
(q+1)\ell \sum_{i=1}^{r}\delta_i
+ q\ell\!\sum_{i=r+1}^{\vert{\mathbf{P}}\vert}\delta_i.
\]
Because $\delta_i\delta_j=\delta_i$ if $i=j$, and $0$ otherwise, we have
\[
\mathcal{L}^2=
(q+1)^2\ell^2\sum_{i=1}^{r}\delta_i
+ q^2\ell^2\sum_{i=r+1}^{\vert{\mathbf{P}}\vert}\delta_i
\]
as all cross-products vanish, and thus
\[
\overline{\mathcal{L}^2}=
(q+1)^2\ell^2\sum_{i=1}^{r}\overline{\delta_i}
+ q^2\ell^2\sum_{i=r+1}^{\vert{\mathbf{P}}\vert}\overline{\delta_i}
= (q+1)^2\ell^2\frac{r}{\vert{\mathbf{P}}\vert}
+ q^2\ell^2\frac{\vert{\mathbf{P}}\vert-r}{\vert{\mathbf{P}}\vert}
= \frac{\ell^2}{\vert{\mathbf{P}}\vert}
(r + 2qr + q^2\vert{\mathbf{P}}\vert).
\]
This allows us to compute the variance of $\mathcal{L}$, as follows:
\begin{align*}
\sigma_{\mathcal{L}}^2
&= \overline{\mathcal{L}^2} - (\overline{\mathcal{L}})^2\\
&= \frac{\ell^2}{\vert{\mathbf{P}}\vert}
\big(r + 2qr + q^2\vert{\mathbf{P}}\vert\big)
- \frac{\vert\mathbf{O}\vert^2}{\vert\mathbf{P}\vert^2}\ell^2\\
&= \frac{\ell^2}{\vert{\mathbf{P}}\vert^2}
\left[r\vert{\mathbf{P}}\vert + 2qr\vert{\mathbf{P}}\vert
+ q^2\vert{\mathbf{P}}\vert^2 
- (q\vert{\mathbf{P}}\vert+r)^2\right]\\
&= \frac{\ell^2 r(\vert{\mathbf{P}}\vert - r)}
{\vert{\mathbf{P}}\vert^2}
\le \frac{\ell^2}{4},
\end{align*}
with equality if and only if $\mathbf{P}=2r$.
We note that in the particular case where $r=0$, we indeed retrieve
the null variance result.

As before, we are now testing the respective results of the
Grapevine algorithm with both original criterion \texttt{6} (left) and
our modified criterion \texttt{6'}, for the same case as above, except
that now $\mathbf{P}=16^2$, which results in $r=16\neq0$ (and $q=39$).
Because the algorithm endowed with the original criterion has already
not been able to obtain an optimal distribution in the easier $r=0$
case, it is safe to expect that it will also not attain an optimal one
in this more complicated case. But it is interesting to test whether
the algorithm equipped with alternate criterion instead still performs
optimally in the case.
With the same initial seeding of the sampler as before, we obtained
the following results, compared against the optimal statistics
computed from the preceding theoretical results:
\begin{center}
\begin{tabular}{lcccc}
\hline
statistic & \textbf{optimal} & iteration
& criterion \texttt{6} & criterion \texttt{6'} \\
\hline\hline
\multirow{2}{*}{$\min{\mathcal{L}}$}
&\multirow{2}{*}{$\bf39$}
&$0$ &\multicolumn{2}{c}{$22$} \\
&&$4$ &$\bf39$    &$\bf39$ \\\hline
\multirow{2}{*}{$\max{\mathcal{L}}$}
&\multirow{2}{*}{$\bf40$}
&$0$ &\multicolumn{2}{c}{$57$} \\
&&$4$ &$45$   &$\bf40$ \\\hline
\multirow{2}{*}{$\sigma_{\mathcal{L}}$}
&\multirow{2}{*}{$\bf0.2421$}
&$0$ &\multicolumn{2}{c}{$5.928$} \\
&&$4$ &$0.5192$  &$\bf0.2421$ \\\hline
\multirow{2}{*}{$\mathcal{I}_\mathcal{L}$}
&\multirow{2}{*}{$\bf0.024$}
&$0$ &\multicolumn{2}{c}{$0.4592$} \\
&&$4$ &$0.152$  &$\bf0.024$ \\\hline
\end{tabular}
\end{center}
We thus observe that once again, while (as expected) the original
algorithm fails to discover an optimal distribution, our version
incorporating modified criterion \texttt{6'} succeeds at doing so.

We therefore propose that the two object iso-load cases (with $r=0$
and with $r\neq0$ be systematically used as baseline test cases for
all distributed load balancers.
Furthermore, we also suggest to use the theoretical results not only
in the idealized iso-load case, but also be used as approximations of
the expected load-balancing results when the objects are of relatively
homogenous sizes.

